\documentclass{report}
\usepackage{amsmath}
\usepackage{graphicx}
\usepackage[dvipsnames]{xcolor}
\usepackage{tikz}
\usetikzlibrary{calc}
\usepackage{anyfontsize}
\usepackage{sectsty}


\input{preamble}
\input{macros}
\input{letterfonts}

\begin{document}

\pagestyle{empty}

\begin{tikzpicture}[overlay,remember picture]

% Background color
\fill[
black!2]
(current page.south west) rectangle (current page.north east);

% Rectangles
\shade[
left color=Dandelion, 
right color=Dandelion!40,
transform canvas ={rotate around ={45:($(current page.north west)+(0,-6)$)}}] 
($(current page.north west)+(0,-6)$) rectangle ++(9,1.5);

\shade[
left color=lightgray,
right color=lightgray!50,
rounded corners=0.75cm,
transform canvas ={rotate around ={45:($(current page.north west)+(.5,-10)$)}}]
($(current page.north west)+(0.5,-10)$) rectangle ++(15,1.5);

\shade[
left color=lightgray,
rounded corners=0.3cm,
transform canvas ={rotate around ={45:($(current page.north west)+(.5,-10)$)}}] ($(current page.north west)+(1.5,-9.55)$) rectangle ++(7,.6);

\shade[
left color=orange!80,
right color=orange!60,
rounded corners=0.4cm,
transform canvas ={rotate around ={45:($(current page.north)+(-1.5,-3)$)}}]
($(current page.north)+(-1.5,-3)$) rectangle ++(9,0.8);

\shade[
left color=red!80,
right color=red!80,
rounded corners=0.9cm,
transform canvas ={rotate around ={45:($(current page.north)+(-3,-8)$)}}] ($(current page.north)+(-3,-8)$) rectangle ++(15,1.8);

\shade[
left color=orange,
right color=Dandelion,
rounded corners=0.9cm,
transform canvas ={rotate around ={45:($(current page.north west)+(4,-15.5)$)}}]
($(current page.north west)+(4,-15.5)$) rectangle ++(30,1.8);

\shade[
left color=RoyalBlue,
right color=Emerald,
rounded corners=0.75cm,
transform canvas ={rotate around ={45:($(current page.north west)+(13,-10)$)}}]
($(current page.north west)+(13,-10)$) rectangle ++(15,1.5);

\shade[
left color=lightgray,
rounded corners=0.3cm,
transform canvas ={rotate around ={45:($(current page.north west)+(18,-8)$)}}]
($(current page.north west)+(18,-8)$) rectangle ++(15,0.6);

\shade[
left color=lightgray,
rounded corners=0.4cm,
transform canvas ={rotate around ={45:($(current page.north west)+(19,-5.65)$)}}]
($(current page.north west)+(19,-5.65)$) rectangle ++(15,0.8);

\shade[
left color=OrangeRed,
right color=red!80,
rounded corners=0.6cm,
transform canvas ={rotate around ={45:($(current page.north west)+(20,-9)$)}}] 
($(current page.north west)+(20,-9)$) rectangle ++(14,1.2);

% Year
\draw[ultra thick,gray]
($(current page.center)+(5,2)$) -- ++(0,-3cm) 
node[
midway,
left=0.25cm,
text width=5cm,
align=right,
black!75
]
{
{\fontsize{25}{30} \selectfont \bf CLASS \\[10pt] NOTES}
} 
node[
midway,
right=0.25cm,
text width=6cm,
align=left,
orange]
{
{\fontsize{72}{86.4} \selectfont 2020}
};

% Title
\node[align=center] at ($(current page.center)+(0,-5)$) 
{
{\fontsize{60}{72} \selectfont {{Differential Equation}}} \\[1cm]
{\fontsize{16}{19.2} \selectfont \textcolor{orange}{ \bf Rohit Raj Karki }}\\[3pt]
Kathmandu University\\[3pt]
Dhulikhel};
\end{tikzpicture}

\newpage% or \cleardoublepage
% \pdfbookmark[<level>]{<title>}{<dest>}
\pdfbookmark[section]{\contentsname}{toc}
\tableofcontents
\pagebreak

\chapter{First Order Differential Equations}
\section{Linear Differential Equation}
A differential equation of form \begin{equation}y'+p(x)y = r(x)
\end{equation}
where 
\begin{align*}
      & p,q\text{ are functions of x alone or constants.}                                                                        
\end{align*}  
 \parinf
 \textbf{Characterstics Features}
\begin{itemize}
\item If q = 0, the equation (1.1) becomes $y'+p(x)y = r(x) $ is homogeneous; otherwise it is
non-homogeneous.
\end{itemize}

\vspace{0.5cm}
\qs{}{Solve the following equation $y'+p(x)y = r(x)$.}
\solve{
\begin{align*}
\Rightarrow & y' = -p(x)y \\
\Rightarrow & \int_{}{} \frac{dy}{y} = \int{}{} -pdx\\
\Rightarrow & ln|y| = \int_{}{} -pdx +c\\
\Rightarrow & y = e^{\int_{}{} pdx} \cdot e^{c_1}\\
\Rightarrow & y=c_2\cdot e^{\int_{}{} pdx}\\
\end{align*}
}
\qs{}{
Solve the following equation $y' + py = r$
}
\solve{
$${
\frac{dy}{dx} = py -r =0
}$$
$${
dy + (py -r)dx = 0 
}$$
\begin{equation}
(py-r)dx + 1dy = 0 
\end{equation}
\begin{center}
Comparing with $Pdx + Qdy = 0$ \\
P = py-r\\
Q = 1\\
\end{center}
$${
\frac{\partial P}{\partial y} = P_y = p, \frac{\partial Q}{\partial x} = Q_y = 0
}$$
\begin{center}
Eqn (1.2) is note exact \\
Let F = F(x) be the integrating factor of eqn (1.2)\\
Then $F = e^{\int_{}{} R(x)dx}$
\end{center}
where 
\begin{center}
$R(x) = \frac{1}{Q}\cdot(P_y-Q_x)$
\end{center}
$${
R(x)=p
}$$
$${F = e^{int_{}{} pdx}
}$$
\begin{center}
Multiplying eqn (1.1) with integrating factor
\end{center}
\begin{align*}
\Rightarrow & e^{\int_{}{}pdx}(y' + py) = re^{\int_{}{}pdx}\\
\Rightarrow & (ye^{\int_{}{}pdx})' = re^{\int_{}{}pdx}\\
\Rightarrow & ye^{\int_{}{}pdx} = \int{}{}re^{\int_{}{}pdx} + C\\
\Rightarrow & y = e^{\int{}{}-pdx}\Big[\int_{}{}re^{\int_{}{}pdx} + C\Big]
\end{align*}
where
\begin{center}
h = $\int_{}{} pdx$\\
The equation is not exact.
\end{center}
}

\qs{}{Solve the following equation $y'-y = e^{2x}$.}
\solve{
\begin{align*}
y&=e^{-x}\Big[\int_{}{}re^{x}dx + c\Big]\\
&=e^{-x}\Big[re^{x} + c\big]
\end{align*}
}

\qs{}{Solve the following equation $x^{3}y' + 3x^{2}y = \frac{1}{x}$}
\solve{}


\section{Bernoulli Equation}
\dfn{}{The equation of the following form is called Bernoulli equation \begin{equation}y' +p(x)y = r(x)y^a
\end{equation} 
If a = 0 or a = 1, equation (1.3) is linear otherwise equation is non linear.\\
}
\ex{}{
Consider the non linear equation
\begin{equation}
y' + py = ry^a
\end{equation} 
Let $u(x) = [y(x)]^{1-a}$
\begin{equation}
u' =(1-a) y^{-a} \cdot y'
\end{equation} 
}
\section{Newton's Law of Cooling}
\thm{}{
The temperature rate of change of the body over respective to time is
\begin{align*}
& \frac{dT}{dt}  \propto (T-T_0)
\end{align*}
where 
\begin{center}
T = f(t)
\end{center}
\begin{align*}
\Rightarrow & \frac{dT}{dt} = k(T-T_0)\\
\Rightarrow & \int_{}{} \frac{dT}{T-T_0} = \int_{}{} kdt\\
\Rightarrow & ln|T-T_0| = kt + c^*\\
\Rightarrow & T = T_0 +ce^{kt}
\end{align*}
}

\qs{}{
A body temperature T is instantly 200\degree C is immersed in a liquid when temperature $T_0$ is constantly.
}
\solve{
We have, \\
\begin{align*}
& \frac{dT}{dt} = k(T-T_0)\\
& \text{initial temperature}(T_0) = 100\degree C\\
& \text{at, t = 0, T = 200\degree C }\\
\Rightarrow & 200 = 100 + ce^{kt}\\
\Rightarrow & ce^{kt} = 100\\
& \therefore c = 100
\end{align*}
Now,
\begin{align*}
& \text{at t = 1 min, T = 100\degree C}\\
\Rightarrow & 150 = 100 + ce^{kt}\\
\Rightarrow & 50 = 100e^{kt}\\
\Rightarrow & e^k = \frac{1}{2}\\
\therefore & k = ln|\frac{1}{2}|
\end{align*}
Now, 

Temperature at 2 min is 
\begin{align*}
T & = T_0 + ce^{kt}\\
& = 100 + 100ce^{2ln|\frac{1}{2}|}\\
& = 100(1 + ce^{ln|\frac{1}{2}|^{2}})\\
& = 100(1 + \frac{1}{4})\\
& = 125\degree C
\end{align*}
}

\section{Electrical Circuits}
\begin{itemize}
\item Ohm's Law
\dfn{}{
Voltage drop across resistor is directly proportional to current.
$$
E_R \propto T
$$
$$
E_R = IR
$$
}
\item Henry Law
\dfn{}{
Voltage drop across inductor is directly proportional to rate of change of current.
\begin{align*}
V_L & \propto \frac{dI}{dt}\\
V_L & = L\frac{dI}{dt}
\end{align*}
}
\item Capcitor Law
\dfn{}{
\begin{align*}
E_e & \propto charge(Q)\\
E_e & = \frac{1}{C} * Q
\end{align*}
}
\item Kirchoff's Law
\dfn{}{
      The algebraic sum of Voltage drop around the charge loop across the closed loop equals zero.
\begin{align*}
\sum IR = 0
\end{align*}
}
\end{itemize}





\chapter{Second Order Differential Equation}
\section{Second Order Linear Differential Equation:}
A differential equation of form 
\begin{align*}
y'' + p(x)y' + q(x)y = r
\end{align*}
\begin{align}
\Rightarrow y'' + py' + qy = r
\end{align}

\nt{
It is called second order differential equation because it has maximum two degree and it has two solutions namely $y_1 \text{and} y_2$.
}

\qs{}{Show that the solutions of the following equation $y'' - y =0$ are $y_1 = e^{-x} , y_2 = e^{x}$.}

\solve{
Let the solution of the equation is $e^{x}$ then \\
$$
y'' = e^{x} and y = e^{x}
$$
\begin{align*}
\Rightarrow & e^{x} - e^{x} = 0 \\
\therefore & 0 =0 \text{(True)} 
\end{align*}

\hspace{0.7 cm}
Let the solution of the equation is $e^{-x}$ then \\
$$
y'' = e^{-x} and y = e^{-x}
$$
\begin{align*}
\Rightarrow & e^{-x} - e^{-x} = 0 \\
\therefore & 0 =0 \text{(True)} 
\end{align*}
}


\thm{}{
      The fundamental theorem of Homogeneous Equation(Superposition Principle or Linearity Principle)\\ \\
      Statement: If $y_1$ and $y_2$ be the solutions of the differential equation $y''+ py' + qy = 0$, then $ y = c_1y_1 + c_2y_2 $.\\
      Proof: Substituting $y = c_1y_1 + c_2y_2 $, we find
      \begin{align*}
      y' = c_1y_1' + c_2y_2'\\
      y'' = c_1y_1'' + c_2y_2''\\
      \end{align*}
      \hspace*{1cm}Substituting y and its derivative in eqn \\
      \begin{align*}
      & (c_1y_1'' + c_2y_2'') + p(c_1y_1' + c_2y_2') + q(c_1y_1 + c_2y_2) = 0\\
      \Rightarrow & c_1y'' + c_2y'' + pc_1y_1' + pc_2y_2' +qc_1y_1 + c_2y_2 = 0\\
      \Rightarrow & c_1(y_1'' + py_1' + qy_1) + c_2(y_2'' + py_2' + qy_2) = 0\\
      \therefore & 0 = 0 (True)
      \end{align*}
      }
\nt{This theorem is only applicable to the homogeneous equation.}
\end{document}
